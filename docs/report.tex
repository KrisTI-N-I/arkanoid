%%% Для сборки выполнить 2 раза команду: pdflatex <имя файла>

\documentclass[a4paper,12pt]{article}

\usepackage{ucs}
\usepackage[utf8x]{inputenc}
\usepackage[russian]{babel}
%\usepackage{cmlgc}
\usepackage{graphicx}
\usepackage{listings}
\usepackage{xcolor}
\usepackage{titlesec}
%\usepackage{courier}


\makeatletter
\renewcommand\@biblabel[1]{#1.}
\makeatother

\newcommand{\myrule}[1]{\rule{#1}{0.4pt}}
\newcommand{\sign}[2][~]{{\small\myrule{#2}\\[-0.7em]\makebox[#2]{\it #1}}}

% Поля
\usepackage[top=20mm, left=30mm, right=10mm, bottom=20mm, nohead]{geometry}
\usepackage{indentfirst}

% Межстрочный интервал
\renewcommand{\baselinestretch}{1.50}

% ------------------------------------------------------------------------------
% minted
% ------------------------------------------------------------------------------
% \usepackage{minted}


% ------------------------------------------------------------------------------
% tcolorbox / tcblisting
% ------------------------------------------------------------------------------
\usepackage{xcolor}
\definecolor{codecolor}{HTML}{FFC300}

\usepackage{tcolorbox}
% \tcbuselibrary{most,listingsutf8,minted}

\tcbset{tcbox width=auto,left=1mm,top=1mm,bottom=1mm,
right=1mm,boxsep=1mm,middle=1pt}

% \newtcblisting{myr}[1]{colback=codecolor!5,colframe=codecolor!80!black,listing only, 
% minted options={numbers=left, style=tcblatex,fontsize=\tiny,breaklines,autogobble,linenos,numbersep=3mm},
% left=5mm,enhanced,
% title=#1, fonttitle=\bfseries,
% listing engine=minted,minted language=r}

%%%%%%%%%%%%%%%%%%%%%%%%%%%%%%%%%%%%%%%

\begin{document}

%%%%%%%%%%%%%%%%%%%%%%%%%%%%%%%
%%%                         %%%
%%% Начало титульного листа %%%

\thispagestyle{empty}
\begin{center}


    \renewcommand{\baselinestretch}{1}
    {\large
        {\sc Петрозаводский государственный университет\\
            Институт математики и информационных технологий\\
            Кафедра информатики и математического обеспечения
        }
    }

\end{center}


\begin{center}
    %%%%%%%%%%%%%%%%%%%%%%%%%
    %
    % Раскомментируйте (уберите знак процента в начале строки)
    % для одной из строк типа направления  - бакалавриат/
    % магистратура и для одной из
    % строк Вашего направление подготовки
    %
    Направление подготовки бакалавриата \\
    % 01.03.02 --- Прикладная математика и информатика \\
    % 09.03.02 --- Информационные системы и технологии \\
    09.03.04 --- Программная инженерия \\
    %%%%%%%%%%%%%%%%%%%%%%%%%
\end{center}

\vfill

\begin{center}

    {\normalsize
        Отчет о проектной работе по курсу <<Основы информатики и программирования>>}
    \medskip

    %%% Название работы %%%
    {\Large \sc {Разработка приложения \\ <<Арканоид>> }} \\
\end{center}

\medskip

\begin{flushright}
    \parbox{11cm}{%
        \renewcommand{\baselinestretch}{1.2}
        \normalsize
        Выполнила:\\
        % Выполнили:\\
        %%% ФИО студента %%%
        студентка 1 курса группы 22107
        \begin{flushright}
            Барсукова Кристина Андреевна \sign[подпись]{4cm}
        \end{flushright}

        %%% Второй участник %%%
        % студента 1 курса группы 2210X
        % \begin{flushright}
        % 	И. О. Фамилия \sign[подпись]{4cm}
        % \end{flushright}

        %%%%%%%%%%%%%%%%%%%%%%%%%
        % девушкам применять "Выполнила" и "студентка"
        %%%%%%%%%%%%%%%%%%%%%%%%%
    }
\end{flushright}

\vfill

\begin{center}
    \large
    Петрозаводск --- 2021
\end{center}

%%% Конец титульного листа  %%%
%%%                         %%%
%%%%%%%%%%%%%%%%%%%%%%%%%%%%%%%

%%%%%%%%%%%%%%%%%%%%%%%%%%%%%%%%
%%%                          %%%
%%% Содержание               %%%

\newpage

\tableofcontents

%%% Содержание              %%%
%%%                         %%%
%%%%%%%%%%%%%%%%%%%%%%%%%%%%%%%

%%%%%%%%%%%%%%%%%%%%%%%%%%%%%%%%
%%%                          %%%
%%% Введение                 %%%

%%% В введении Вы должны описать предметную область, с которой связана %%%
%%% Ваша работа, показать её актуальность, вкратце определить цель     %%%
%%% разработки					       %%%


\newpage
\section*{Введение}
\addcontentsline{toc}{section}{Введение}

Цель проекта: разработать игру <<Арканоид>> на С++ и QML (QtQuick).

Задачи проекта:
%%% Пример создания списков %%%
\begin{enumerate}
    \item Разработать графический интерфейс на QML
    \item Реализовать классы C++ для описания объектов игры
    \item Реализовать обработку столкновения объектов
    \item Реализовать логику игры
\end{enumerate}

%%%                          %%%
%%%%%%%%%%%%%%%%%%%%%%%%%%%%%%%%

%%%%%%%%%%%%%%%%%%%%%%%%%%%%%%%
%%% Требования к приложению %%%
% \newpage
\section{Требования к приложению}
\begin{itemize}
    \item Приятный графический интерфейс
    \item Функционал игры <<Арканоид>> (шарик с гравитацией, управляемая платформа и блоки, которые можно сбивать)
    \item Возможность управления игрой с клавиатуры
    \item Возможность перезапуска игры
\end{itemize}
% \subsection{Подраздел}

%%%                                     %%%
%%%%%%%%%%%%%%%%%%%%%%%%%%%%%%%%%%%%%%%%%%%

%%%%%%%%%%%%%%%%%%%%%%%%%%%%%%%%%%%%%%%%%%%
%%%                                     %%%
%%% Проектирование приложения           %%%
% \newpage
\section{Проектирование приложения}

Модули приложения:

\begin{enumerate}
    \item game -- С++ модуль, реализующий основной функционал игры.

          Переменная QList<QPair<bool, QRect> > blocks хранит положение и состояние каждого блока (true -- блок активен, false -- блок сломан).
    \item panel -- С++ модуль, который описывает объект <<платформа>> (координаты и размер).
    \item ball -- С++ модуль, который описывает объект <<шарик>> (координаты, размер и скорость).
    \item collision -- С++ модуль, функции для обработки столкновения объектов. Использованы функции QRect.intersects() и QRect.contains().
    \item Ball.qml -- часть графического интерфейса, шарик.
    \item Block.qml -- часть графического интерфейса, блок. При создании выбирается случайный цвет.
    \item Panel.qml -- часть графического интерфейса, платформа.
    \item main.qml -- основной интерфейс. Поле для игры создаётся при помощи Repeater. Для оповещения о конце игры используется QtQuick.Dialogs.
\end{enumerate}

%%%                          %%%
%%%%%%%%%%%%%%%%%%%%%%%%%%%%%%%%

%%%%%%%%%%%%%%%%%%%%%%%%%%%%%%%%
%%%                          %%%
%%% Реализация приложения    %%%
% \newpage
\section{Реализация приложения}

Для реализации игры были использованы языки <<C++>> и <<QML>>. В приложении активно используются стандартные билиотеки Qt -- QRect, QPoint, QSize, QTimer и др..
\begin{itemize}
    \item Количество модулей: 8
    \item Количество классов: 4
    \item Количество C++ функций: 29
    \item Количество QML функций: 7
    \item Количество строк C++ кода: 261
    \item Количество строк QML кода: 258
\end{itemize}

%%%                          %%%
%%%%%%%%%%%%%%%%%%%%%%%%%%%%%%%%

%%%%%%%%%%%%%%%%%%%%%%%%%%%%%%%%
%%%                          %%%
%%% Заключение               %%%

\newpage
\section*{Заключение}
\addcontentsline{toc}{section}{Заключение}
В итоге получено приложение-игра <<Арканоид>>, удалось реализовать все запланированные функции. Реализована простая физика для игровых объектов. Получен новый опыт создания компьютерных игр и работы с QtQuick.

Финальный вид интерфейса:
\begin{center}
    \includegraphics[scale=0.8]{images/Screenshot_20210601_233616.png}
\end{center}

%%%                          %%%
%%%%%%%%%%%%%%%%%%%%%%%%%%%%%%%%

%%%%%%%%%%%%%%%%%%%%%%%%%%%%%%%%
%%%                          %%%
%%% Приложение               %%%

\newpage
\appendix
%\section*{Приложение}
%\addcontentsline{toc}{section}{Приложение}
%\titleformat{\section}[display]
%  {\normalfont\Large\bfseries}
%  {Приложение\ \thesection}
%  {0pt}{\Large\centering}
%\renewcommand{\thesection}{\Asbuk{section}}

%%%                          %%%
%%%%%%%%%%%%%%%%%%%%%%%%%%%%%%%%
\end{document}

